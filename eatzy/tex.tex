🧠 أشهر اختصارات Bootstrap ومعناها:
✅ الحالات (States):
active 👉 تعني العنصر مفعّل (غالبًا يكون لونه أبيض أو مميز).

disabled 👉 تعني العنصر غير نشط أو غير قابل للنقر.
-------------------------------------------------------------------
🔵 المسافات (Spacing):
🟡 الاتجاهات:
الاختصار	المعنى
t	top (أعلى)
b	bottom (أسفل)
s	start (بداية - يسار في RTL)
e	end (نهاية - يمين في RTL)
x	أفقيًا (يمين ويسار)
y	عموديًا (أعلى وأسفل)
--------------------------------------------------------------------
🟤 مفيدة جدًا:
text-center : توسيط النص

d-flex : جعل العنصر Flexbox

justify-content-between : توزيع العناصر بمسافة بينية

align-items-center : توسيط عمودي داخل Flex

(
|  |
| ---------------------------------- | ------------------------------- |
| `margin-left: auto;` أو `ms-auto`  | يدفع العنصر إلى **اليمين**      |
| `margin-right: auto;` أو `me-auto` | يدفع العنصر إلى **اليسار**      |
)

-----------------------------------------------------------------------------
لثبات الnav في الصفحة 
position-fixed w-100
position-static top-0
وحدة منهن صح


https://dev.w3.org/html5/html-author/charref
للايقونات html


--------------------------------------------------------

g-4:gap:للمسافات 



<!-- Footer -->
<footer class="footer-menu mb-4">
  <div class="container d-flex justify-content-center align-items-center flex-column gap-5  pt-5">
    
    <img src="./assets/image/footer/footer.jpg" alt="">
    
    <div class="d-flex align-items-center gap-2">
      <i class="fa-solid fa-house"></i>
      <h5 class="mb-0">No: 58 A, East Madison Street, Baltimore, MD, USA 4508</h5>
    </div>
    
    <div class="d-flex align-items-center gap-2">
      <i class="fa-solid fa-phone"></i>
      <h5 class="mb-0">0000 - 123 - 456789</h5>
    </div>
    
    <div class="d-flex align-items-center gap-2">
      <i class="fa-solid fa-envelope"></i>
      <h5 class="mb-0">info@example.com</h5>
    </div>

    <div class="d-flex align-items-center gap-2">
      <i class="fa-brands fa-twitter"></i>
      <i class="fa-brands fa-facebook"></i>
      <i class="fa-brands fa-google-plus"></i>
      <i class="fa-brands fa-tumblr"></i>
    </div>
    
  </div>
</footer>